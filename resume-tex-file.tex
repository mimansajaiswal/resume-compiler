%%%%%%%%%%%%%%%%%%%%%%%%%%%%%%%%%%%%%%%%%%%%%%%%%%%%%%%%%%%%%%%%%%%%%%%%
%%%%%%%%%%%%%%%%%%%%%% Simple LaTeX CV Template %%%%%%%%%%%%%%%%%%%%%%%%
%%%%%%%%%%%%%%%%%%%%%%%%%%%%%%%%%%%%%%%%%%%%%%%%%%%%%%%%%%%%%%%%%%%%%%%%

%%%%%%%%%%%%%%%%%%%%%%%%%%%%%%%%%%%%%%%%%%%%%%%%%%%%%%%%%%%%%%%%%%%%%%%%
%% NOTE: If you find that it says                                     %%
%%                                                                    %%
%%                           1 of ??                                  %%
%%                                                                    %%
%% at the bottom of your first page, this means that the AUX file     %%
%% was not available when you ran LaTeX on this source. Simply RERUN  %%
%% LaTeX to get the ``??'' replaced with the number of the last page  %%
%% of the document. The AUX file will be generated on the first run   %%
%% of LaTeX and used on the second run to fill in all of the          %%
%% references.                                                        %%
%%%%%%%%%%%%%%%%%%%%%%%%%%%%%%%%%%%%%%%%%%%%%%%%%%%%%%%%%%%%%%%%%%%%%%%%

%%%%%%%%%%%%%%%%%%%%%%%%%%%% Document Setup %%%%%%%%%%%%%%%%%%%%%%%%%%%%

% Don't like 10pt? Try 11pt or 12pt
\documentclass[10pt]{article}

% The automated optical recognition software used to digitize resume
% information works best with fonts that do not have serifs. This
% command uses a sans serif font throughout. Uncomment both lines (or at
% least the second) to restore a Roman font (i.e., a font with serifs).
%\usepackage{times}
%\renewcommand{\familydefault}{\sfdefault}

% This is a helpful package that puts math inside length specifications
\usepackage{calc}
\usepackage{comment}
\usepackage{array}      % +++++++++++++++++++++++++++++++++++++++
\usepackage{ifthen}     % for conditional statements
\usepackage[absolute]{textpos}  % for absolute positioning

% Simpler bibsection for CV sections
% (thanks to natbib for inspiration)
\makeatletter
\newlength{\bibhang}
\setlength{\bibhang}{1em} %1em}
\newlength{\bibsep}
 {\@listi \global\bibsep\itemsep \global\advance\bibsep by\parsep}
\newenvironment{bibsection}
        {\begin{list}{}{
        \setlength{\leftmargin}{0em}
       \setlength{\itemsep}{\bibsep}
       \setlength{\parsep}{\z@}
        \setlength{\partopsep}{0pt}
        \setlength{\topsep}{0pt}}}
        {\end{list}\vspace{-.6\baselineskip}}           % +++++++++++++++++++++++++++++++++++++++
\makeatother

% Layout: Single column format
%\reversemarginpar

%
%         PAPER SIZE, PAGE NUMBER, AND DOCUMENT LAYOUT NOTES:
%
% The next \usepackage line changes the layout for CV style section
% headings as marginal notes. It also sets up the paper size as either
% letter or A4. By default, letter was used. If A4 paper is desired,
% comment out the letterpaper lines and uncomment the a4paper lines.
%
% As you can see, the margin widths and section title widths can be
% easily adjusted.
%
% ALSO: Notice that the includefoot option can be commented OUT in order
% to put the PAGE NUMBER *IN* the bottom margin. This will make the
% effective text area larger.
%
% IF YOU WISH TO REMOVE THE ``of LASTPAGE'' next to each page number,
% see the note about the +LP and -LP lines below. Comment out the +LP
% and uncomment the -LP.
%
% IF YOU WISH TO REMOVE PAGE NUMBERS, be sure that the includefoot line
% is uncommented and ALSO uncomment the \pagestyle{empty} a few lines
% below.
%

%% Use these lines for letter-sized paper
\usepackage[paper=letterpaper,
            %includefoot, % Uncomment to put page number above margin
            margin=0.65in]{geometry}

%% Use these lines for A4-sized paper
%\usepackage[paper=a4paper,
%            %includefoot, % Uncomment to put page number above margin
%            marginparwidth=30.5mm,    % Length of section titles
%            marginparsep=1.5mm,       % Space between titles and text
%            margin=25mm,              % 25mm margins
%            includemp]{geometry}

%% More layout: Get rid of indenting throughout entire document
\setlength{\parindent}{0in}

\usepackage[shortlabels]{enumitem}

%% Reference the last page in the page number
%
% NOTE: comment the +LP line and uncomment the -LP line to have page
%       numbers without the ``of ##'' last page reference)
%
% NOTE: uncomment the \pagestyle{empty} line to get rid of all page
%       numbers (make sure includefoot is commented out above)
%
\usepackage{fancyhdr,lastpage}
\pagestyle{fancy}
%\pagestyle{empty}      % Uncomment this to get rid of page numbers
\fancyhf{}\renewcommand{\headrulewidth}{0pt}
\cfoot{\arabic{page} of \protect\pageref*{LastPage}}

% Finally, give us PDF bookmarks
\usepackage{color,hyperref}
\definecolor{darkblue}{rgb}{0.0,0.0,0.3}
\hypersetup{colorlinks,breaklinks,
            linkcolor=darkblue,urlcolor=darkblue,
            anchorcolor=darkblue,citecolor=darkblue}

%%%%%%%%%%%%%%%%%%%%%%%% End Document Setup %%%%%%%%%%%%%%%%%%%%%%%%%%%%


%%%%%%%%%%%%%%%%%%%%%%%%%%% Helper Commands %%%%%%%%%%%%%%%%%%%%%%%%%%%%

% The title (name) with a horizontal rule under it
% (optional argument typesets an object right-justified across from name
%  as well)
%
% Usage: \makeheading{name}
%        OR
%        \makeheading[right_object]{name}
%
% Place at top of document. It should be the first thing.
% If ``right_object'' is provided in the square-braced optional
% argument, it will be right justified on the same line as ``name'' at
% the top of the CV. For example:
%
%       \makeheading[\emph{Curriculum vitae}]{Your Name}
%
% will put an emphasized ``Curriculum vitae'' at the top of the document
% as a title. Likewise, a picture could be included:
%
%   \makeheading[\includegraphics[height=1.5in]{my_picutre}]{Your Name}
%
% the picture will be flush right across from the name.
\newcommand{\makeheading}[2][]%
        {\begin{center}
             {\Large \bfseries #2}
             \ifthenelse{\equal{#1}{}}{}{\\[0.1\baselineskip] {\small #1}}
             \\[0.2\baselineskip]
             \rule{\textwidth}{1pt}
         \end{center}}

% The section headings
%
% Usage: \section{section name}
\renewcommand{\section}[1]{\pagebreak[3]%
    \hyphenpenalty=10000%
    \vspace{1\baselineskip}%
    \phantomsection\addcontentsline{toc}{section}{#1}%
    \noindent{\normalsize\bfseries\MakeUppercase{#1}}%
    \vspace{0.5\baselineskip}\par}

% An itemize-style list with lots of space between items
\newenvironment{outerlist}[1][\enskip\textbullet]%
        {\begin{itemize}[#1,leftmargin=*]}{\end{itemize}%
         \vspace{-.6\baselineskip}}

% An environment IDENTICAL to outerlist that has better pre-list spacing
% when used as the first thing in a \section
\newenvironment{lonelist}[1][\enskip\textbullet]%
        {\begin{list}{#1}{%
        \setlength{\partopsep}{0pt}%
        \setlength{\topsep}{0pt}}}
        {\end{list}\vspace{-.6\baselineskip}}

% An itemize-style list with little space between items
\newenvironment{innerlist}[1][\enskip\textbullet]%
        {\begin{itemize}[#1,leftmargin=*,parsep=0pt,itemsep=0pt,topsep=0pt,partopsep=0pt]}
        {\end{itemize}}

% An environment IDENTICAL to innerlist that has better pre-list spacing
% when used as the first thing in a \section
\newenvironment{loneinnerlist}[1][\enskip\textbullet]%
        {\begin{itemize}[#1,leftmargin=*,parsep=0pt,itemsep=0pt,topsep=0pt,partopsep=0pt]}
        {\end{itemize}\vspace{-.6\baselineskip}}

% To add some paragraph space between lines.
% This also tells LaTeX to preferably break a page on one of these gaps
% if there is a needed pagebreak nearby.
\newcommand{\blankline}{\quad\pagebreak[3]}
\newcommand{\halfblankline}{\quad\vspace{-0.5\baselineskip}\pagebreak[3]}

% Uses hyperref to link DOI
\newcommand\doilink[1]{\href{http://dx.doi.org/#1}{#1}}
\newcommand\doi[1]{doi:\doilink{#1}}

% For \url{SOME_URL}, links SOME_URL to the url SOME_URL
\providecommand*\url[1]{\href{#1}{#1}}
% Same as above, but pretty-prints SOME_URL in teletype fixed-width font
\renewcommand*\url[1]{\href{#1}{\texttt{#1}}}

% Allow better line breaking for long text
\sloppy

% For \email{ADDRESS}, links ADDRESS to the url mailto:ADDRESS
\providecommand*\email[1]{\href{mailto:#1}{#1}}
% Same as above, but pretty-prints ADDRESS in teletype fixed-width font
%\renewcommand*\email[1]{\href{mailto:#1}{\texttt{#1}}}

%\providecommand\BibTeX{{\rm B\kern-.05em{\sc i\kern-.025em b}\kern-.08em
%    T\kern-.1667em\lower.7ex\hbox{E}\kern-.125emX}}
%\providecommand\BibTeX{{\rm B\kern-.05em{\sc i\kern-.025em b}\kern-.08em
%    \TeX}}
\providecommand\BibTeX{{B\kern-.05em{\sc i\kern-.025em b}\kern-.08em
    \TeX}}
\providecommand\Matlab{\textsc{Matlab}}

%%%%%%%%%%%%%%%%%%%%%%%% End Helper Commands %%%%%%%%%%%%%%%%%%%%%%%%%%%

%%%%%%%%%%%%%%%%%%%%%%%%% Begin CV Document %%%%%%%%%%%%%%%%%%%%%%%%%%%%

\begin{document}

% Last updated in top right corner - positioned absolutely
\begin{textblock*}{3in}[1,0](\paperwidth-0.3in,0.25in)
\raggedleft\footnotesize Last Updated on \today
\end{textblock*}

\begin{center}
{\Large \bfseries Mimansa Jaiswal}
\\[0.2\baselineskip]
% \small
(734) 747-0283 $\mid$
\href{mailto:mimansa.jaiswal@gmail.com}{\texttt{mimansa.jaiswal@gmail.com}} $\mid$
\href{mailto:mimansa@umich.edu}{\texttt{mimansa@umich.edu}} $\mid$
\href{https://mimansajaiswal.github.io/}{\texttt{Website}} $\mid$
\href{https://twitter.com/MimansaJ}{@MimansaJ} $\mid$
\href{https://www.linkedin.com/in/mimansajaiswal/}{LinkedIn}
\\[0.2\baselineskip]
\rule{\textwidth}{1pt}
\end{center}
\vspace{0.3\baselineskip}

\section{Interests}

My work focuses on \textbf{LLM post-training and evaluation}, with emphasis on identifying and diagnosing model failures through systematic error analysis and adversarial testing \textbf{\textit{(Failure Analysis)}}, applying alignment techniques including RLHF/RLAIF, DPO/GRPO, and reward modeling to improve model behavior and capabilities \textbf{\textit{(Post-Training)}}, designing automated evaluation frameworks and synthetic data generation pipelines to assess model performance across fine-grained skills and edge cases \textbf{\textit{(Evaluation \& Data Augmentation)}}, incorporating human feedback, preferences, and domain expertise into iterative model development workflows \textbf{\textit{(Human-in-the-Loop ML)}}, and developing interpretable model analysis techniques to understand and improve model reliability for real-world deployment \textbf{\textit{(Model Interpretability \& Deployment)}}.
%%%%%%%%%%%%%%%%%%%%%%%%%%%%%%%%%%%%%%%%%%%%%%%%%%%%%%%%%%%%%%%%%%%%%%%%%%%%%%%
%%%%%%%%%%%%%%%%%%%%%%%%%%%%%%%%%%%%%%%%%%%%%%%%%%%%%%%%%%%%%%%%%%%%%%%%%%%%%%%

\section{Experience}
\href{https://ai.meta.com/}{\textbf{Meta (MSL Org)}} | Research Scientist
 \hfill {Feb 2025 to \textit{present}} \\
\vspace{-.1in}
\begin{innerlist}
\item[]
    %\vspace{-.1in}
    \begin{innerlist}
    \item[] Work on \href{https://www.meta.ai/}{Meta AI's} Product Capabilities team to develop and fine-tune large language models for usecases (agentic tool calling, personalized support and tutoring) using post-training techniques including RLHF/RLAIF (with and without checklists), DPO/GRPO, and reward modeling to improve production models.
    \item[] Design and implement synthetic data generation pipelines and automated evaluation systems, including rubric-based judges, preference data collection workflows, and quality assessment frameworks to measure and improve model performance.
    \end{innerlist}
\end{innerlist}
\vspace{.05in}


\href{https://norm.ai/}{\textbf{Norm AI}} | Research Engineer
\hfill {Oct 2023 to Oct 2024} \\
\vspace{-.1in}
\begin{innerlist}
\item[]
    %\vspace{-.1in}
    \begin{innerlist}
    \item[] Developed domain-specific question answering and retrieval-augmented generation (RAG) systems for legal applications, using synthetic data augmentation to expand training datasets, create evaluation benchmarks, and improve model coverage.
    \item[] Designed agentic LLM simulation frameworks to model human group perception and consensus-building for subjective law interpretation and policy evaluation.
    \end{innerlist}
\end{innerlist}
\vspace{.05in}


\href{https://allennlp.org/}{\textbf{Allen NLP, Allen Institute for AI}} | Research Intern
\hfill {Sep 2021 to Dec 2021} \\
\vspace{-.1in}
\begin{innerlist}
\item[]
    %\vspace{-.1in}
    \begin{innerlist}
    \item[] Developed interpretable evaluation frameworks for GPT-3 and other language models that decompose complex tasks into component skills, creating benchmarks for multi-aspect quality assessment.
    \item[] Applied these evaluation methodologies to measure fine-grained model capabilities and identify failure modes in natural language understanding tasks.
    %Manager: \href{https://www.anamarasovic.com/}{Ana Marasovic}
    \end{innerlist}
\end{innerlist}
\vspace{.05in}

\href{https://ai.facebook.com/research/NLP}{\textbf{NLP Team, Facebook AI Research (FAIR)}} | Research Intern \hfill {May 2021 to Aug 2021} \\
\vspace{-.1in}
\begin{innerlist}
\item[]
    %\vspace{-.1in}
    \begin{innerlist}
    \item[] Identified and analyzed systematic failure patterns in Natural Language Inference (NLI) through adversarial testing and error analysis.
    \item[] Developed correction strategies and data augmentation techniques to improve model robustness on challenging NLI examples, experimenting with the CTRL model.
    %Manager: \href{https://wp.nyu.edu/adinawilliams/}{Adina Williams}, Peers: \href{http://scottyih.org/}{Scott Yih}, \href{https://www.pedro.ai/}{Pedro Rogieruez}
    \end{innerlist}
\end{innerlist}
\vspace{.05in}

\vspace{.05in}

\href{https://ai.facebook.com/research/conversational-ai/}{\textbf{Conversation AI, Facebook Research}} | Research Intern
\hfill {May 2020 to Aug 2020} \\
\vspace{-.1in}
\begin{innerlist}
\item[]
    %\vspace{-.1in}
    \begin{innerlist}
    \item[] Developed interpretable user satisfaction metrics for conversational AI systems by integrating human knowledge and behavioral signals, training weakly supervised hierarchical label models.
    \item[] Created evaluation frameworks that combine quantitative metrics with qualitative human feedback to assess dialogue quality.
    %Manager: \href{https://sites.google.com/view/beirami}{Ahmad Beirami}, Peers: \href{https://shanemoon.com/}{Shane}, \href{https://satwikkottur.github.io/}{Satwik}, \href{https://chinnadhurai.github.io/}{Chinnadhurai}
    \end{innerlist}
\end{innerlist}

% \vspace{.05in}

% \href{http://chai.eecs.umich.edu}{\textbf{CHAI Lab, University of Michigan}},
% Ann Arbor, MI \hfill {Sept 2017 to present} \\
% \vspace{-.1in}
% \begin{innerlist}
% \item[]
%     \begin{innerlist}
%     \item[] Robust and interpretable systems for social signal processing using natural language understanding and speech processing
%     \item[] Position: Graduate Student Research Assistant (GSRA); Advisor: \href{http://web.eecs.umich.edu/~emilykmp/}{Prof. Emily Mower Provost}
%     \end{innerlist}
% \end{innerlist}

% \vspace{.05in}

% \href{http://www.ietdavv.edu.in/}{\textbf{Institute of Engineering and Technology}},
% Indore, India \hfill {July 2016 to May 2017} \\
% \vspace{-.1in}
% \begin{innerlist}
% \item[]
%     \begin{innerlist}
%     \item[] Linguistic markers of mental health issues and generating automod reply conditioned on affect
%     \item[] Position: B.E. Thesis Project; Advisor: Prof. G.L Prajapati
%     \end{innerlist}
% \end{innerlist}

% \vspace{.05in}


% \href{https://sentic.net/team/}{\textbf{SenticNet | Computational Intelligence Lab | NTU}},
% Singapore \hfill {Apr 2016 to July 2016} \\
% \vspace{-.1in}
% \begin{innerlist}
% \item[]
%     \begin{innerlist}
%     \item[] Multi-modal deception detection using automated feature set.
%     \item[] Detecting empathetic and non-empathetic responses in comments, and where they are needed.
%     \item[] Position: Undergraduate Research Assistant; Advisor: \href{http://sentic.net/erikcambria/}{Prof. Erik Cambria}
%     \end{innerlist}
% \end{innerlist}

% \vspace{.05in}


% \href{http://academiasinicanlplab.github.io/}{\textbf{NLP and Sentiment Analysis Lab | Academia Sinica}},
% Taiwan \hfill {Dec 2015 to Feb 2016} \\
% \vspace{-.1in}
% \begin{innerlist}
% \item[]
%     \begin{innerlist}
%     \item[] Emotion Push: Automatically conveying emotion in instant messages and notifications.
%     \item[] GMMs for sentence recommendation system for learning difference in synonyms for ESL learners.
%     \item[] Position: Undergraduate Research Assistant; Advisor: \href{http://www.iis.sinica.edu.tw/pages/lwku/index_en.html}{Prof. Lun-Wei Ku}
%     \end{innerlist}
% \end{innerlist}

% \vspace{.05in}

% \href{https://www.iiitd.ac.in}{\textbf{Indraprastha Institute of Information Technology Delhi}},
% Delhi, India \hfill {May 2015 to July 2015} \\
% \vspace{-.1in}
% \begin{innerlist}
% \item[]
%     \begin{innerlist}
%     \item[] Finding the optimal conditions under which people are willing to share rides.
%     \item[] Developed an algorithm to optimize route matching for cab-pooling.
%      \item[] Position: Undergraduate Research Assistant; Advisor: \href{http://faculty.iiitd.ac.in/~praveshb/}{Prof. Pravesh Biyani}
%     \end{innerlist}
% \end{innerlist}

% \vspace{.05in}


% \textbf{Zootout},
% Indore, India \hfill {Jan 2015 to April 2015} \\
% \vspace{-.1in}
% \begin{innerlist}
% \item[] \emph{Undergraduate Technical Assistant} \\
%     \vspace{-.1in}
%     \begin{innerlist}
%     \item[] Aspect based sentiment analysis on hotel and restaurant reviews.
%     \item[] Sentiment analysis to find the overall inclination of review for desired tags.
%     \item[] Mentor: Sumay Dubey
%     \end{innerlist}
% \end{innerlist}

\vspace{.05in}

% \newpage



%%%%%%%%%%%%%%%%%%%%%%%%%%%%%%%%%%%%%%%%%%%%%%%%%%%%%%%%%%%%%%%%%%%%%%%%%%%%%%%
%%%%%%%%%%%%%%%%%%%%%%%%%%%%%%%%%%%%%%%%%%%%%%%%%%%%%%%%%%%%%%%%%%%%%%%%%%%%%%%
¸% \section{Relevant Coursework}

% \emph{@}UMich: Machine Learning, Introduction to AI, Computer Vision,
% Computational Complexity, \\Advanced Compilers, and Advanced Computer Networks.
% \\\\
% \emph{@}UC: Probability \& Random Processes, Optimization Methods,
% Digital Signal Processing, \\Multivariate Calculus, Differential Equations,
% and Linear Algebra.

%%%%%%%%%%%%%%%%%%%%%%%%%%%%%%%%%%%%%%%%%%%%%%%%%%%%%%%%%%%%%%%%%%%%%%%%%%%%%%%
%%%%%%%%%%%%%%%%%%%%%%%%%%%%%%%%%%%%%%%%%%%%%%%%%%%%%%%%%%%%%%%%%%%%%%%%%%%%%%%
% \section{In Progress Work}
% \vspace{-.1275in}
% \begin{flushleft}
% \begin{bibsection}

% \item[] \textbf{Environmental and Synthetic Noise which is either Perceptible or Imperceptible by Humans, All Break Emotion Recognition.} \\
% \textbf{Mimansa Jaiswal}, Emily Mower Provost

% \end{bibsection}
% \end{flushleft}

\section{Works in Progress}
\vspace{-.1275in}
\begin{flushleft}
\begin{bibsection}

\item[] \textbf{CAPSTONE: Capability Assessment Protocol for Systematic Testing Of Natural language models' Expertise}. \textbf{Mimansa Jaiswal}.
\\\href{https://mimansajaiswal.github.io/posts/capstone/}{Research Notes}

% \item[] \textbf{Controlled Evaluation of Explanations: What Might Have Influenced Your Model Explanation Efficacy Evaluation?}\\
% %In \textit{EMNLP 2020}.\\
% \textbf{Mimansa Jaiswal}, Minxue Niu

% \item[] \textbf{An Interpretation Framework for Explaining Multimodal Models Trained for Computational Paralinguistic Tasks.}\\
% %In \textit{EMNLP 2020}.\\
% \textbf{Mimansa Jaiswal}

\item[] \textbf{Designing Interfaces for Delivering and Obtaining Generation Explanation Annotations}. \textbf{Mimansa Jaiswal}.
\\\href{https://human-in-loop-explanation-annotation.vercel.app/}{Demo} and \href{https://github.com/mimansajaiswal/hil-text-annotation}{Repo}

% \item[] \textbf{Investigating the Effect of Noise on Human Perception and Machine Classification of Emotion.} \\
% In \textit{Interspeech 2020}. \\
% \textbf{Mimansa Jaiswal}, Emily Mower Provost


%\\\textbf{Abstract accepted} to \textit{Women in Machine Learning Workshop, co-located with NeurIPS 2019}.

\end{bibsection}
\end{flushleft}

\section{Submitted Publications}
\vspace{-.1275in}
\begin{flushleft}
\begin{bibsection}
\item[]\href{https://arxiv.org/abs/2408.16961}{\textbf{The Future of Open Human Feedback.}} Shachar Don-Yehiya, Ben Burtenshaw, Ramon Fernandez Astudillo, Cailean Osborne, \textbf{Mimansa Jaiswal}, Tzu-Sheng Kuo, Wenting Zhao, Idan Shenfeld, Andi Peng, Mikhail Yurochkin, Atoosa Kasirzadeh, Yangsibo Huang, Tatsunori Hashimoto, Yacine Jernite, Daniel Vila-Suero, Omri Abend, Jennifer Ding, Sara Hooker, Hannah Rose Kirk, Leshem Choshen.
\item \textbf{Temperature Zero Isn't the Best Choice For Accuracy in Legal Reasoning}. \textbf{Mimansa Jaiswal}, Scott Worland, John Nay.
\item[]\href{https://arxiv.org/abs/2405.14782}{\textbf{Lessons from the Trenches on Reproducible Evaluation of Language Models.}} Stella Biderman, Hailey Schoelkopf, Lintang Sutawika,…(and 12 others) \textbf{Mimansa Jaiswal} …(and 10 others).
\item[] \href{https://mimansajaiswal.github.io/papers-pdf/sally_anne_workshop.pdf}{\textbf{Sally-Anne False Belief Test for LLMs}}. \textbf{Mimansa Jaiswal}.
\item[] \href{https://mimansajaiswal.github.io/papers-pdf/qualitative-gender-workshop.pdf}{\textbf{Qualitatively Studying Gender Biases in LLMs}}. \textbf{Mimansa Jaiswal}.
\item[] \href{https://mimansajaiswal.github.io/notes/case-of-llm-evals.html}{\textbf{Assessing Large Language Models: A Comprehensive Survey and Critical Analysis of Evaluation Metrics and Methodologies}}. \textbf{Mimansa Jaiswal}.
% \item[] \textbf{Investigating the Effect of Noise on Human Perception and Machine Classification of Emotion.} \\
% In \textit{Interspeech 2020}. \\
% \textbf{Mimansa Jaiswal}, Emily Mower Provost


%\\\textbf{Abstract accepted} to \textit{Women in Machine Learning Workshop, co-located with NeurIPS 2019}.

\end{bibsection}
\end{flushleft}


\section{Accepted Publications}
\vspace{-.1275in}

\begin{flushleft}
\begin{bibsection}
\item[] \href{https://arxiv.org/pdf/2408.17026}{\textbf{From Text to Emotion: Unveiling the Emotion Annotation Capabilities of LLMs}}. In \textit{Interspeech 2024}. Minxue Niu, \textbf{Mimansa Jaiswal}, Emily Mower Provost.

\item[] \href{https://www.isca-archive.org/interspeech_2023/niu23b_interspeech.pdf}{\textbf{Capturing Mismatch between Textual and Acoustic Emotion Expressions for Mood Identification in Bipolar Disorder}}. In \textit{Interspeech 2023}. Minxue Niu, Amrit Romana, \textbf{Mimansa Jaiswal}, Melvin McInnis, Emily Mower Provost.

\item[] \href{https://www.isca-speech.org/archive/pdfs/interspeech_2022/perez22_interspeech.pdf}{\textbf{Mind the gap: On the value of silence representations to lexical-based speech emotion recognition}}. In \textit{Interspeech 2022}. Matthew Perez, \textbf{Mimansa Jaiswal}, Minxue Niu, Cristina Gorrostieta, Matthew Roddy, Kye Taylor, Reza Lotfian, John Kane, Emily Mower Provost.

\item[] \href{https://arxiv.org/pdf/2104.08792.pdf}{\textbf{Human-Centered Metric Design to Promote Generalizable and Debiased Emotion Recognition.}} At Text as Data (TADA) conference 2021. \textbf{Mimansa Jaiswal}, Emily Mower Provost.

\item[] \href{https://arxiv.org/pdf/2104.08806.pdf}{\textbf{Noise Based Augmentation of Emotion Datasets: What's ideal and What Isn't?.}} In \textit{ACL-SRW 2020}. \textbf{Mimansa Jaiswal}, Emily Mower Provost.

\item[] \href{https://aclanthology.org/2020.lrec-1.187.pdf}{\textbf{MuSE: Multimodal Stressed Emotion Dataset.}} In \textit{Conference on Language Resources and Evaluation (LREC) 2020}. \textbf{Mimansa Jaiswal$^*$}, CP Bara, Yuanhang Luo, Rada Mihalcea, Mihai Burzo, Emily Mower Provost.

\item[] \href{https://arxiv.org/pdf/1910.13212.pdf}{\textbf{Privacy Enhanced Multimodal Neural Representations for Emotion Recognition.}} In \textit{AAAI Conference on Artificial Intelligence (AAAI) 2020}. In \textit{Human Centered Machine Learning (HCML), Privacy in Machine Learning (PriML) Workshops in NeuRIPS 2019}. \textbf{Mimansa Jaiswal}, Emily Mower Provost.

\item[] \href{https://arxiv.org/pdf/1908.08979.pdf} {\textbf{Controlling for Confounders in Multimodal Emotion Classification via Adversarial Learning.}} In \textit{International Conference on Multimodal Interaction (ICMI) 2019}. \textbf{Mimansa Jaiswal}, Zakaria Aldeneh, Emily Mower Provost.

\item[] \href{https://www.isca-speech.org/archive/Interspeech_2019/pdfs/1878.pdf}{\textbf{Identifying Mood Episodes Using Dialogue Features from Clinical Interviews.}} In \textit{Interspeech 2019}. Zakaria Aldeneh, \textbf{Mimansa Jaiswal}, Michael Picheny, Melvin McInnis, Emily Mower Provost.

\item[] \href{https://arxiv.org/pdf/1903.11672.pdf}{\textbf{MuSE-ing on the Impact of Utterance Ordering On Crowdsourced Emotion Annotations.}} In \textit{IEEE International Conference on Acoustics, Speech, and Signal Processing (ICASSP) 2019}. \textbf{Mimansa Jaiswal}, Zakaria Aldeneh, Cristian-Paul Bara, Yuanhang Luo, Mihai Burzo, Rada Mihalcea, Emily Mower Provost.

\item[] \href{https://arxiv.org/pdf/1806.10658.pdf}{\textbf{The PRIORI Emotion Dataset: Linking Mood to Emotion Detected In-the-Wild.}} In \textit{Interspeech 2018}. Soheil Khorram, \textbf{Mimansa Jaiswal}, John Gideon, Melvin McInnis, Emily Mower Provost.

% \item[] \href{https://arxiv.org/pdf/1903.05210.pdf}{
% \textbf{“Hang In There:" Lexical and Visual Analysis to Identify Posts Warranting Empathetic Responses.}}\\
% In \textit{The Florida Artificial Intelligence Research Society Conference (FLAIRS) 2017}. \\
% \textbf{Mimansa Jaiswal}, Sairam Tabibu, Erik Cambria

% \item[] \href{https://arxiv.org/pdf/1903.04484.pdf}{
% \textbf{The Truth and Nothing but the Truth: Multimodal Analysis for Deception Detection.}}
% \\
% In \textit{International Conference on Data Mining Workshops (ICDMW) 2016}. \\
% \textbf{Mimansa Jaiswal}, Sairam Tabibu, Rajiv Bajpai

% \item[]
% \textbf{Contextual Text-mining Approach for Teacher Feedback.}
% \\
% In \textit{Extended Abstracts, Information Systems Research and Teaching,  2014}. \\
% \textbf{Mimansa Jaiswal},  Vinshi Vanvat, Manasi Tiwari

\end{bibsection}
\end{flushleft}
%\vspace{.15in} % don't need this if this is the end of a page

%%%%%%%%%%%%%%%%%%%%%%%%%%%%%%%%%%%%%%%%%%%%%%%%%%%%%%%%%%%%%%%%%%%%%%%%%%%%%%%
%%%%%%%%%%%%%%%%%%%%%%%%%%%%%%%%%%%%%%%%%%%%%%%%%%%%%%%%%%%%%%%%%%%%%%%%%%%%%%%

% \newpage
%%%%%%%%%%%%%%%%%%%%%%%%%%%%%%%%%%%%%%%%%%%%%%%%%%%%%%%%%%%%%%%%%%%%%%%%%%%%%%%
%%%%%%%%%%%%%%%%%%%%%%%%%%%%%%%%%%%%%%%%%%%%%%%%%%%%%%%%%%%%%%%%%%%%%%%%%%%%%%%


\section{Education}

\href{http://www.umich.edu}{\textbf{University of Michigan}},
Ann Arbor, MI \hfill {Sept 2017 to Aug 2023} \\
\vspace{-.1in}
\begin{innerlist}
\item[] \textbf{\href{https://rackham.umich.edu/discover-rackham/announcing-the-2022-2023-barbour-scholars/}{Awarded Barbour Fellowship Amongst Over 1k+ Applicants}}
\item[] Ph.D. in Computer Science and Engineering
\item[] Computational Human Analysis and Integration \textit{(CHAI)} Lab
\item[] \textbf{Advisor}: \href{http://web.eecs.umich.edu/~emilykmp}{Prof. Emily Provost}
\item[] \textbf{Thesis}: Implicit Design Choices and Their Impact on Emotion Recognition Model Development and Evaluation
\end{innerlist}
\vspace{.05in}

\href{http://www.umich.edu}{\textbf{University of Michigan}},
Ann Arbor, MI \hfill {Sept 2017 to May 2019} \\
\vspace{-.1in}
\begin{innerlist}
\item[] M.S. in Computer Science and Engineering, GPA: 3.87/4.00
\item[] \textbf{Coursework:} Natural Language Processing, Advanced Artificial Intelligence, Bayesian Inference
\end{innerlist}
\vspace{.05in}

\href{http://www.ietdavv.edu.in/}{\textbf{Institute of Engineering and Technology}},
Indore, India \hfill {July 2013 to May 2017} \\
\vspace{-.1in}
\begin{innerlist}
\item[] B.E. in Computer Science Engineering, GPA: 3.7/4.00
\item[] \textbf{Coursework:} Information Retrieval, Machine Learning, Alogirthm Design and Analysis

\end{innerlist}

\section{Honors and Activities}

\begin{itemize}[leftmargin=0pt,parsep=3pt,itemsep=0pt,topsep=0pt,partopsep=0pt]
\item[] \textbf{Reviewing:} ICDMW, ICMI, ACII, CHI, CSCW, *ACL, AAAI, NeurIPS \textit{(2019-present)}.
\item[] \textbf{\href{https://rackham.umich.edu/discover-rackham/announcing-the-2022-2023-barbour-scholars/}{Awarded Barbour Fellowship Amongst Over 1k+ Applicants}}
\item[] Graduate Student Instructor for Affective Computing Course (Winter 2020).
\item[] Student Representative in CSE Faculty Hiring Committee (Academic Year 2020-21).
\item[] Invited Speaker for PyCon 2016 Singapore \href{https://www.youtube.com/watch?v=rhVhR22t2IE&t=0s&list=PLMrPHnpQRHYZcOoyqcTMpFR1t-WLJAxb9&index=7}{[video]}, PyCon 2016 India \href{https://www.youtube.com/watch?v=A1IKw67vYkE}{[video]}
\item[] Invited to attend CRA-W Grad Cohort Workshop 2018, 2020 [declined]
\item[] National Talent Search Examination Scholarship (Top 0.01\% in India)
%\item[] Zonal Informatics Olympiad conducted by IARCS (Top 1\% in India)
%\item[] Selected for Winter School of Data Analytics 2013 by Indo-German Max Planck Center (Top 1\%)
\end{itemize}

%%%%%%%%%%%%%%%%%%%%%%%%%%%%%%%%%%%%%%%%%%%%%%%%%%%%%%%%%%%%%%%%%%%%%%%%%%%%%%%
%%%%%%%%%%%%%%%%%%%%%%%%%%%%%%%%%%%%%%%%%%%%%%%%%%%%%%%%%%%%%%%%%%%%%%%%%%%%%%%

% \section{Skills and Languages}

% \begin{itemize}[leftmargin=0pt,parsep=3pt,itemsep=0pt,topsep=0pt,partopsep=0pt]
% \item[] Python, Java, C++, English [TOEFL: 118/120], Hindi
% \end{itemize}

%%%%%%%%%%%%%%%%%%%%%%%%%%%%%%%%%%%%%%%%%%%%%%%%%%%%%%%%%%%%%%%%%%%%%%%%%%%%%%%
%%%%%%%%%%%%%%%%%%%%%%%%%%%%%%%%%%%%%%%%%%%%%%%%%%%%%%%%%%%%%%%%%%%%%%%%%%%%%%%

% \section{References}

% Available upon request

\end{document}

%%%%%%%%%%%%%%%%%%%%%%%%%% End CV Document %%%%%%%%%%%%%%%%%%%%%%%%%%%%%

%----------------------------------------------------------------------%
% The following is copyright and licensing information for
% redistribution of this LaTeX source code; it also includes a liability
% statement. If this source code is not being redistributed to others,
% it may be omitted. It has no effect on the function of the above code.
%----------------------------------------------------------------------%
% Copyright (c) 2007, 2008, 2009, 2010, 2011 by Theodore P. Pavlic
%
% Unless otherwise expressly stated, this work is licensed under the
% Creative Commons Attribution-Noncommercial 3.0 United States License. To
% view a copy of this license, visit
% http://creativecommons.org/licenses/by-nc/3.0/us/ or send a letter to
% Creative Commons, 171 Second Street, Suite 300, San Francisco,
% California, 94105, USA.
%
% THE SOFTWARE IS PROVIDED "AS IS", WITHOUT WARRANTY OF ANY KIND, EXPRESS
% OR IMPLIED, INCLUDING BUT NOT LIMITED TO THE WARRANTIES OF
% MERCHANTABILITY, FITNESS FOR A PARTICULAR PURPOSE AND NONINFRINGEMENT.
% IN NO EVENT SHALL THE AUTHORS OR COPYRIGHT HOLDERS BE LIABLE FOR ANY
% CLAIM, DAMAGES OR OTHER LIABILITY, WHETHER IN AN ACTION OF CONTRACT,
% TORT OR OTHERWISE, ARISING FROM, OUT OF OR IN CONNECTION WITH THE
% SOFTWARE OR THE USE OR OTHER DEALINGS IN THE SOFTWARE.
%----------------------------------------------------------------------%
